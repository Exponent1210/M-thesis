
\section{計算の詳細}

\subsection{1. レーザー光による励起子の生成}
実験に則した励起光のビームサイズを仮定し、時間ステップ内に生成すべき数のパラ励起子をそのビーム体積内にランダムに生成させた。なお、その初速の運動エネルギーは12 meVとし、初速度の方向はランダムに割り当てた。

\subsection{2. 寿命による励起子の消失}
寿命による励起子のロスを扱うには、時間ステップ $\Delta t$ あたりに何個の励起子を消失させるかを求める必要がある。ここでは寿命の効果を次式で表す。

\begin{equation}
    \frac{dN}{dt} = -\frac{N}{\tau} \tag{7.62}
\end{equation}

ただし、$N$ は励起子数、$\tau$ は励起子の寿命を示す。時間ステップ $\Delta t$ あたりに加えるべき励起子数は以下のようになる。

\begin{equation}
    \Delta N = -N \frac{\Delta t}{\tau} \tag{7.63}
\end{equation}

この式に基づき、$N$ 個の励起子からランダムに励起子を選び、消失させ、次の時間ステップへ進む。

\subsection{3. トラップポテンシャルの導入}
トラップポテンシャルは、Hertz型経験則に基づく計算と実験との比較をもとに、4次式の多項式をフィットさせることで計算した。運動方程式は次のように表される。

\begin{equation}
    m_{ex} \frac{d^2 r}{dt^2} = -\frac{dV(r)}{dr} \tag{7.64}
\end{equation}

これを1階微分の形に分けると、以下のようになる。

\begin{align}
    m_{ex} \frac{dv}{dt} &= -\frac{dV(r)}{dr} \tag{7.65} \\
    \frac{dr}{dt} &= v \tag{7.66}
\end{align}

さらにこれを離散化すると次の式が得られる。

\begin{align}
    m_{ex} \Delta v &= \frac{dV(r)}{dr} \tag{7.67} \\
    v &= \Delta t \alpha \tag{7.68}
\end{align}

\subsection{4. 励起子格子相互作用}
励起子とLAフォノンの相互作用の計算には、以下の手順を用いる。

\begin{itemize}
    \item (a) ランダムに励起子を選び、その波数ベクトル $K$ を抽出する。
    \item (b) フォノンの波数ベクトル $q$ をランダムに選ぶ。
    \item (c) $q \leq K$ となる場合、その遷移を採択する。
    \item (d) $f_q, F_q$ は適切な確率密度を持つように設定する。
    \item (e) 採択された場合、フォノン放出では $K - q$、吸収では $K + q$ に変更する。
\end{itemize}

\section{遷移の行列要素とエネルギー保存則}

フォノン吸収と放出に関する遷移の行列要素は次のように与えられる。

\begin{align}
    |M|^2 &= \frac{\hbar^2 q}{2 \rho V} (1 + f_{k+q}) F_q \tag{7.69} \\
    |M|^2 &= \frac{\hbar^2 q}{2 \rho V} (1 + f_{k-q}) (1 + F_q) \tag{7.70}
\end{align}

エネルギー保存則は以下の通り。

\begin{align}
    \frac{\hbar^2 K^2}{2 m_{ex}} + \hbar \omega_q &= \frac{\hbar^2 (K + q)^2}{2 m_{ex}} \tag{7.71} \\
    \frac{\hbar^2 K^2}{2 m_{ex}} &= \hbar \omega_q + \frac{\hbar^2 (K - q)^2}{2 m_{ex}} \tag{7.72}
\end{align}

運動量保存則は次の式で表される。

\begin{align}
    q &= 2K \cos (\pi - \theta) + \frac{2 m_{ex} v}{\hbar} \tag{7.73} \\
    q &= 2K \cos \theta - \frac{2 m_{ex} v}{\hbar} \tag{7.74}
\end{align}

\section{シミュレーションの効率化と計算手法}

パラ励起子の輻射再結合を考慮し、直接遷移の運動量保存則を満たす励起子を見つけた場合にその位置と速度を記録する方式を用いた。実際の計算では、保存則を厳密に満たす確率は非常に低いため、±2\% の許容幅を設けた。

さらに、仮想的なスリット位置における保存則の適用により、エネルギー・空間分解スペクトルの再現を試みた。シミュレーションでは、$10^9$ 個の励起子の全追跡は難しいため、1つの仮想粒子に $5 \times 10^4$ 個の励起子を代表させて計算を効率化した。

時間ステップは $\Delta t = 1 \, \text{ps}$ とし、1 µs まで計算を行った。空間サイズは $(500 \, \mu\text{m})^3$ とし、32 × 32 × 32 のセルに分割し、並列計算を行った。詳細なプログラムは付録Bに掲載している。

% chapter1.tex
\newpage
\section{序論}
\renewcommand{\thefigure}{\thesection-\arabic{figure}}
\subsection{研究の背景}
直接遷移型半導体である亜酸化銅結晶の励起子はその長寿命の性質から長年励起子BECの候補として注目されてきた。
\subsection{研究の目的}
励起子のオルソパラ転換レートの測定を行った。
応力が印加された亜酸化銅バルク結晶に、黄色系列1sオルソ励起子のフォノンサイドバンド吸収に対応する波長の可視光を照射することで光生成されたと、バルク内で光生成された黄色系列1sオルソ励起子は、より低いエネルギー準位に位置するは黄色系列1sパラ励起子へ転換する。
先行研究では、オルソ励起子からパラ励起子への転換レートは2 K以上の温度領域での報告にとどまり、サブケルビン温度領域では未解明であった。
今回の実験では結晶の冷却に希釈冷凍機を用いて、サブケルビン温度領域でのオルソーパラ転換レートの測定を初めて行い、転換レートの温度依存性を取得した
BECの本質は非対角長距離秩序(超流動性)を持つかどうかである。電子正孔の凝縮体がこれを持つかは自明ではない。
今までにパラ励起子の寿命、トラップ周波数、冷却レート、二体散乱によるロス、オルソパラ転換レートが励起子BECの形成過程に必要なパラメーターである。
サブケルビン温度領域で構成される励起子BECの形成過程の理解には当該温度領域における転換レートの実験的評価とそれによる従来の物理機構の検証が必須となっている。
b
励起子は結晶と相互作用する環境に置かれているため、非平衡解放系の量子コヒーレンスの物理学に新しい知見をもたらすと考えられている。
凝縮体の形成ダイナミクスなども解明されていない点が残されている。
非平衡統計力学の開拓につながることが期待されている。
\subsection{研究の構成}
第2、3章ではボーズ-アインシュタイン凝縮
励起子の背景知識について紹介する。第4章では亜酸化銅結晶の背景知識について紹介する。第章ではオルソパラ転換の観測の実験についてまとめる。第章ではまとめと今後の展望について述べる。
